\documentclass{article}
\usepackage{amsmath}
    \title{Numerical Solutions to the Schrodinger Equation}
    \author{Yash Gandhi}
    
    \begin{document}
    \maketitle
    
    \section{Project Proposal}
    The majority of problems in quantum mechanics can be linked to the Schrodinger equation
    \begin{equation} \label{eq:1}
        -\frac{\hbar^2}{2m} \nabla^2\Psi(\vec{x})_n + V(\vec{x})\Psi(\vec{x})_n = E_n\Psi(\vec{x})_n.
    \end{equation}
    This equation describes the motion of particles defined by quantum mechanical properties as
    opposed to classical mechanics. Although, as the number of particles increases, the coulomb potential
    term becomes problematic and makes this differential an incredibly difficult one to solved
    analytically. To overcome this, there are quite a few algorithms that have allowed researchers
    to solve more difficult and more complex multi-body problems using numerical approximations.
    In this project, I will explore numerical approximations to some of the most commonly studied 
    problems: particle in a box, particle in an infinite well, and the hydrogen atom model (harmonic
    oscillator). These three problems have been analytically solved, so I will be able to compare 
    the approximation to the exact solution. These results allow me to explore convergence rates, 
    order of accuracies, and explore interesting phenomenon such as quantum tunneling. Equation \ref{eq:1}
    shows the Schrodinger equation in multiple dimensions, but I will mainly be working with the 1D or 
    2D time independent Schrodinger equation. When we are only concerned about one dimension 
    equation \ref{eq:1} reduces to
    \begin{equation}\label{eq:2}
        -\frac{\hbar^2}{2m} \frac{d^2\psi(x)_n}{dx^2} + V(x)\psi(x)_n = E_n\psi(x). 
    \end{equation}
    For some cases, $V(x) = 0$, this simplifies to a second order ODE and is easy to solve with 2nd order 
    finite difference methods. In the case of the hydrogen atom, the potential 
    \begin{equation}\label{eq:3}
        V(r) = k_e\frac{q_1q_2}{r} = \frac{e^2}{4\pi\varepsilon_0 r}.
    \end{equation} 
    is a function of position making the problem more difficult and calls for more complex numerical 
    methods such as using algebraic solvers.
    
    



    \end{document}